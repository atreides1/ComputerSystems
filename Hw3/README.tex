\documentclass[12pt]{article}

\usepackage{amsmath}
\usepackage{amssymb}

\usepackage{color}
\usepackage[total={6in,8in}]{geometry}
\usepackage{amsthm}

\usepackage{enumitem}
\usepackage{centernot}

\newcommand{\N}{\mathbb{N}}
\newcommand{\Z}{\mathbb{Z}}
\newcommand{\Q}{\mathbb{Q}}
\newcommand{\R}{\mathbb{R}}
\newcommand{\C}{\mathbb{C}}
\newcommand{\sn}{\mathfrak{S}}
\newcommand{\ve}{\varepsilon}
\setlength{\parindent}{1cm}


\author{Marika Swanberg}

\begin{document}
Marika Swanberg and Jillian James, Homework 2, \today
\bigskip 

\section{Basic Cache Operations}
We implemented our cache using an unordered\_map. Our unordered\_map is of type \texttt{std::unordered\_map<std::string, std::tuple<val\_type, index\_type, index\_type>>} For each entry in the unorderedmap, there is a key that hashes to a position where to a tuple is stored (tuple is implemented with the tuple library). The tuple has three elements: a val\_type value, an index\_type size, and an index\_type age. Value is a pointer to a block of allocated memory, size is the size of this memory block, and age determines the order in which the eviction policy will delete this value. The functions \texttt{get()}, \texttt{set()}, \texttt{del()}, use this unordered\_map functionality. \textbf{Check top block comment on cache.cc for compilation/run information.}

\section{Testing}
We have created a test file, test.cc, which tests the functionality described in this README file. In particular, it has ten unit tests that test different aspects of the cache: memory management, get and set functionality, and evictor functionality. The specific purpose of each test is described in a comment preceding each unit test in test.cc.

\section{Resizing, Performance, Collision Resolution}
We took advantage of the $C^{++}$ implementation of an unorderedmap, and used the resizing and collision resolution system that unordered map uses. In particular, unorderedmap uses a hash function and buckets in order to reduce and handle collisions. Resiszing is performed in a way that is similar to $C^{++}$ vectors.


\section{Eviction Policy}
We implemented FIFO in a simple manner. The unordered map maps a key to a tuple containing the a value, a size, and an age. We keep a global
variable named ``newest'', and each time a new entry is added to the unordered map, its age is set to newest and then newest is incremented.

For eviction, we implemented FIFO by looping through the cache to find the oldest entry (the one with the lowest number for age.), and deleting
that.

We will explain why we chose to implement FIFO in this manner: First, we wanted to try to implement an $O(1)$ evictor.
This would mean keeping track of the age and keeping track of the oldest element, and evicting that element.
At first this seemed easy enough to implement, but we ran into a number of difficulties. We thought we would need a separate map where that
map's keys are the ages of each element in our cache, and the values are the keys used to access the elements in our cache. But what should we
do if an element from the middle of the cache was deleted? We would still have to check if the oldest element was still in our cache, and in the
case where it wasn't we would need to loop to find the next oldest value, defeating the point of trying to implement an $O(1)$ evictor.

We were able to find technical solutions to these problems, but they were not very graceful. It involved modifying nearly every function in the
cache implementation. Our second idea for implementing a constant-time evictor was writing a subclass of the cache implementation that was
specifically an FIFO cache, and modifying the \texttt{set()} function (and for the LRU implementation the \texttt{get()} function as well) there.
We would have preferred this method, but we ran into difficulties with inheritance class structure because of our lack of familiarity with $C^{++}
$. While it's not ideal to have an $O(n)$ evictor, it might not be horrible. The evictor is only called if the user has surpassed the cache
capacity. And for an LRU implementation, it may be reasonable, since LRU caches update all of the age bits of each entry every time that any item
is accessed or modified. So, while these implementations might have an $O(1)$ evictor, they have $O(n)$ runtimes in other functions.
(Unfortunatley we did not complete our LRU implementation because we had difficulties getting out ages to update in the \texttt{get()} function
so this implementation is not shown.) In conclusion, it seemed that while \texttt{evictor()} could be sped up to $O(1)$, we did not have the
requisite knowledge to do this in a way that would be both beneficial and isolated enough from the rest of the code. 


\end{document}